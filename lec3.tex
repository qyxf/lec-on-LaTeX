\documentclass{beamer}
\usepackage{ctex}
\setCJKmainfont{思源宋体} % 设置缺省中文字体
\usepackage{fourier}     % 配置西文与数学字体

\hypersetup{pdfcreator=TeX,
pdfpagemode = FullScreen,
}

\mode<presentation>
{
  \usetheme{XJTU}
  \useinnertheme{rounded}
  \usefonttheme{serif}
}

\title[Tech Lec]{钱院学辅技术讲座}
\subtitle{第三讲:\TeX 基础}

\author[xjtu-blacksmith]{黑山雁}

\institute[Xi'an Jiaotong University]
{
  西安交通大学
}

\date{\today}

\subject{技术,讲座}

\AtBeginSubsection[]
{
  \begin{frame}<beamer>{目录}
    \tableofcontents[currentsection,currentsubsection]
  \end{frame}
}

%\beamerdefaultoverlayspecification{<+->}


\begin{document}
\begin{frame}
\maketitle
\end{frame}

\begin{frame}{目录}
    \tableofcontents
\end{frame}

\section{内容、样式与排版}
\subsection{概念陈述}
\subsection{组织作品的原则}
% 1. 内容与样式分离
% 2. 排版不能太烂
% 3. 内容为王
% 4. 低门槛,高质量
\subsection{排版艺术简述}

\section{\TeX{} 是什么}
\subsection{\TeX{} and friends}
\subsection{\TeX 系统简析}
\subsection{第一份\TeX 文档}


\section{你所需要知道的\TeX}
\subsection{语法细节:\TeX 的嘴和胃}
\subsection{盒子、分词、分段、分页}
\subsection{宏定义:黑魔法}
\subsection{数学公式入门}

\end{document}


