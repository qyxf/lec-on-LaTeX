\documentclass{beamer}
\usepackage{ctex}
\setCJKmainfont{思源宋体} % 设置缺省中文字体
\usepackage{fourier}     % 配置西文与数学字体

\hypersetup{pdfcreator=TeX,
pdfpagemode = FullScreen,
}

\mode<presentation>
{
  \usetheme{XJTU}
  \useinnertheme{rounded}
  \usefonttheme{serif}
}

\title[Technical Lecture of QYXF]{钱院学辅技术讲座}
\subtitle{第零讲:总述}

\author[xjtu-blacksmith]{黑山雁}

\institute[Xi'an Jiaotong University]
{
  西安交通大学
}

\date{\today}

\subject{技术,讲座}

\AtBeginSection[]
{
  \begin{frame}<beamer>{目录}
    \tableofcontents[currentsection]
  \end{frame}
}

%\beamerdefaultoverlayspecification{<+->}


\begin{document}
\begin{frame}
\maketitle
\end{frame}

\begin{frame}{目录}
    \tableofcontents
\end{frame}

\section{关于学辅与「技术」}

\subsection{钱院学辅简介}
\begin{frame}{钱院学辅简介}{这不是推广页面}
\begin{description}
    \item[钱院学辅] 全称「西安交通大学钱学森书院学生会学业辅导中心」,是校级学业辅导中
    心在钱学森书院的分支机构。
    \item[目前工作] 线上与线下答疑、\textbf{资料编写}、筹办讲座与交流活动等。
    \item[技术部门] 负责一系列技术手段的测试与维护:\LaTeX{} 排版技术、网站建设、文档下载
    平台……
    \item[GitHub 主页] \url{https://github.com/qyxf}
    \item[信息站] \url{https://qyxf.site}   
\end{description}
\end{frame}



\subsection{我来学什么?}

\begin{frame}{我来学什么?}
\begin{itemize}
    \item 能源、材料、\textbf{信息}
    \item 信息需要整理、分析、呈现!
    \item 发布我的信息:聊天信息、Word 文档、PDF 文档、网站页面、公众号推送……
    \item 三大板块:开源协作、专业排版、网站建设
    \begin{description}
        \item[开源协作] 利用、改进已有成果,不重造轮子
        \item[专业排版] 为你的优质内容锦上添花
        \item[网站建设] 学习最快、最方便的网站搭建方法  
    \end{description}
\end{itemize}
\end{frame}

\subsection{我们的原则}
\begin{frame}{我们的原则}
\begin{enumerate}
    \item 用门槛最低、最为方便的手段实现需求。
    \item 在以上前提下,追求最佳效果、最优体验。
    \item 在以上前提下,注意「内容大于形式」的原则。
\end{enumerate}
\end{frame}

\subsection{给自己定位}
\begin{frame}{给自己定位}
\begin{itemize}
    \item 我是什么专业?将来想干什么?
    \item 我有多少空闲时间?我对什么感兴趣?这对我的主业有无帮助?
    \item 我不是程序员!
\end{itemize}

\begin{alertblock}{学习「技术」的第一原则}
记得自己的主业,牢记自己学习的目的,注意时间与精力的分配。
\end{alertblock}

\end{frame}

\section{内容总述}

\subsection{第一讲:网页与 Markdown}
\begin{frame}{第一讲:网页与 Markdown}
\begin{enumerate}
    \item 网页设计原理:新「三剑客」(HTML、CSS、Javascript)
    \item 为何 Markdown:提高文档编写效率,拓宽文档的应用范围
\end{enumerate}

\begin{block}{结论一}
Markdown 是目前最易于使用的文本标记语言,没有之一。
\end{block}

\end{frame}

\subsection{第二讲:Git 与 GitHub}
\begin{frame}{第二讲:Git 与 GitHub}
\begin{enumerate}
    \item 「开源」探讨:从程序员的圈子之外来看「开源世界」
    \item Git 简介:一切提交,均有备份
    \item 走近 GitHub:欢迎来到「全球最大同性交友网站」!
\end{enumerate}

\begin{block}{结论二}
「开源」将在未来几十年内成为大学生的必修课。
\end{block}

\begin{block}{结论三}
使用 GitHub Pages + Markdown 渲染工具是目前最快、最方便的建站方式。
\end{block}

\end{frame}

\subsection{第三讲:\TeX{} 基础}
\begin{frame}{第三讲:\TeX{} 基础}
\begin{enumerate}
    \item 排版基础:逃离 Office 舒适圈!
    \item \TeX{} 基础:什么叫专业排版!(战术后仰)
\end{enumerate}

\begin{block}{结论四}
排版是一项非常专业的工作;但我们只需要稍加学习,即可远远超越平均水平。
\end{block}
\end{frame}

\subsection{第四讲:\LaTeX{} 入门}
\begin{frame}{第四讲:\LaTeX{} 入门}
\begin{enumerate}
    \item \LaTeX{} 学习新思路:从 Markdown 开始
    \item \LaTeX{} 的基本编写流程与进一步学习:\LaTeX{} 真的需要终身学习!
\end{enumerate}

\begin{block}{结论五}
\[\text{\LaTeX}\approx\text{\TeX}+\text{Markdown}\]
\end{block}
\end{frame}

\subsection{第五讲:\LaTeX{} 提高与排版规范}
\begin{frame}{第五讲:\LaTeX{} 提高与排版规范}
\begin{enumerate}
    \item \LaTeX{} 的宏包:一份宏包,一分效果!
    \item 中西文排版规范:从源头保护大家的眼睛。
    \item 终章:没有尽头的学习与实践之路。
\end{enumerate}

\begin{block}{结论六}
明确的目标、开阔的视野,比技术本身更加重要。
\end{block}
\end{frame}

\section{最后……}
\begin{frame}{最后……}
\begin{block}{讲座的文本资料}
\url{https://qyxf.site/technique/}
\end{block}

\begin{block}{讲座幻灯片下载}
\url{https://qyxf.site/BookHub/007.technical-lecture/}
\end{block}
\end{frame}

\end{document}


