\documentclass{beamer}
\usepackage{ctex}
\setCJKmainfont{思源宋体} % 设置缺省中文字体
\usepackage{fourier}     % 配置西文与数学字体

\hypersetup{pdfcreator=TeX,
pdfpagemode = FullScreen,
}

\mode<presentation>
{
  \usetheme{XJTU}
  \useinnertheme{rounded}
  \usefonttheme{serif}
}

\title[Tech Lec]{钱院学辅技术讲座}
\subtitle{第零讲:总述}

\author[xjtu-blacksmith]{黑山雁}

\institute[Xi'an Jiaotong University]
{
  西安交通大学
}

\date{\today}

\subject{技术,讲座}

\AtBeginSubsection[]
{
  \begin{frame}<beamer>{目录}
    \tableofcontents[currentsection,currentsubsection]
  \end{frame}
}

%\beamerdefaultoverlayspecification{<+->}


\begin{document}
\begin{frame}
\maketitle
\end{frame}

\begin{frame}{目录}
    \tableofcontents
\end{frame}

\section{关于“技术”}
\subsection{学辅的技术宗旨}
% 1. 开源协作,寻求外部支持
% 2. 用最简单、最高效的技术/工作方式
% 3. 内容大于形式
\subsection{给自己定位}
% 我是什么专业?将来想干什么?我想在学辅获得什么?
% 我有多少空闲时间?我对什么感兴趣?这对我的主业有无帮助?
% 我不是程序员!

\section{内容总述}
% 第一讲:网页与 Markdown
% 第二讲:Git 与 GitHub
% 第三讲:TeX 基础
% 第四讲:LaTeX 入门
% 第五讲:LaTeX 提高与排版规范

\end{document}


