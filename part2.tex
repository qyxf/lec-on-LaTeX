\documentclass[10pt]{beamer}

\hypersetup{pdfcreator=TeX,
pdfpagemode = FullScreen,
}

\usepackage{fontspec}
\usepackage[fontset=adobe]{ctex}
\usepackage{metalogo,texnames}
\setmainfont{Cambria}

\mode<presentation>
{
  \usetheme{XJTU}
  \useinnertheme{rounded}
  \usefonttheme{serif}
}

\usepackage{multicol}
\usepackage{indentfirst}

\title[钱学森书院学业辅导中心]{\LaTeX 讲座}
\subtitle{第二部分:\LaTeX 知识清单}

\author[黑山雁]{\texttt{黑山雁 xjtu-blacksmith}}

\institute[Xi'an Jiaotong University]
{\textbf{西安交通大学·钱学森书院学业辅导中心}}

\date{\today}

\subject{LaTeX 讲座}

\begin{document}

\begin{frame}
  \titlepage
\end{frame}

\begin{frame}{目录}
  \tableofcontents
\end{frame}

\section{我该学习哪些内容?}
\begin{frame}{我该学习哪些内容?}{学完是不可能学完的}
\red{\textbf{基本内容:}}
\begin{enumerate}
    \item 基本样式:文档类、页边距、页眉页脚
    \item 文档结构:标题、章节、目录、交叉引用、嵌入、附录等环境
    \item 文本控制:字体切换、行距段距、段落及页面控制、基本文本环境
    \item 图表处理:插图、表格绘制、浮动体使用
    \item 数学公式:各类公式环境、符号输入、定理环境
    \item 基层语法:命令、环境、长度与参数更改、宏包引用、报错识别
\end{enumerate}

\red{\textbf{进阶内容:}}
\begin{enumerate}
    \item 页面设计:盒子、样式表、封面设计
    \item 宏包与文档类编写
    \item 周边功能:\texttt{pgf}/Ti$k$Z,\texttt{beamer}
    \item ……
    \item \LaTeX\&\TeX 底层拆分与设计(\textit{成为大佬})
\end{enumerate}
\end{frame}

\section{基本样式}
\begin{frame}{基本样式}
\begin{itemize}
    \item \red{三种基本文档类}
    \footnote{它们所对应的参数,可参考\texttt{lshort}\textbf{表1.2}。}
    \red{:}\texttt{article},\texttt{book},\texttt{report}
    \item \red{页边距设置:}参数设置
    \footnote{可参考\texttt{lshort}\textbf{图5.1}。}
    ,\texttt{geometry}宏包
    \item \red{页眉页脚设定:}预定义样式
    \footnote{可参考\texttt{lshort}\textbf{表5.5}。}
    ,\texttt{fancyhdr}宏包
\end{itemize}
\end{frame}

\section{文档结构}
\begin{frame}{文档结构}
\begin{block}{可编译的\LaTeX{}文档必须具有的内容}
\begin{enumerate}
    \item 文档类的设定(\texttt{\textbackslash{}documentclass})
    \item \texttt{document}环境
\end{enumerate}
\end{block}
\begin{block}{其他类型的\LaTeX{}代码文档}
\begin{enumerate}
    \item 用于嵌入到可编译文档的分文档(利用\texttt{\textbackslash{}input} 或\texttt{\textbackslash{}include} 命令)
    \item 利用特殊环境代替\texttt{document} 环境(如\texttt{tikzpicture} )
    \item 宏包文件(\texttt{sty})或文档类文件(\texttt{cls})
\end{enumerate}
\end{block}
\begin{alertblock}{\LaTeX{}不止是排版}
注意段落、章节、图表等内容层次与结构的设计!即使原作者在这方面做的不好,排版者也有义务提高印刷品在此方面的质量!

\red{\textbf{好的结构:}}章节层次分明,各节长度适中;段落疏密有致,图表浮动自如。
\end{alertblock}
\end{frame}

\section{文本控制}
\begin{frame}{文本控制}
\begin{enumerate}
    \item \red{字体系统}:字族、字形等概念
    \footnote{可以参考刘海洋《\LaTeX{}入门》P64表\textbf{2.5}。}
    ;对应命令(全局/局部样式
    \footnote{例如,\texttt{\textbackslash{}bf}和\texttt{\textbackslash{}bfseries}是“全局样式命令”,
    而\texttt{\textbackslash{}textbf}则是“局部样式命令”。准确的说,“全局样式命令”应该被称为\textbf{控制序列},是
    \TeX 下的概念;“局部样式命令”可简称为\textbf{命令},必须带参数,是\LaTeX 下的概念。}
    )
    \item \red{文本环境}:三种基本列表,居左/居中/居右环境,\texttt{quote}与
    \texttt{quotation},诗歌环境(\texttt{verse}),抄录环境(\texttt{
    verbatim})
    \item \red{特殊结构}:抄录段(\texttt{\textbackslash{}verb}),脚注(\texttt{\textbackslash{}footnote},
    \texttt{\textbackslash{}footnotemark} + \texttt{\textbackslash{}footnotetext}),交叉引用
    (\texttt{\textbackslash{}label},\texttt{\textbackslash{}ref},\texttt{\textbackslash{}cite})
    \item \red{分段、换行、分页}
\end{enumerate}
\begin{alertblock}{切勿滥用强制换行}
请深入理解\LaTeX 的\textbf{段落机制},牢记“\textit{换行连续,连续换行才是分段}”,不要联想到所见即所得软件中随意换行的机制!
\end{alertblock}
\end{frame}

\section{图表处理}
\begin{frame}{图表处理}
\red{\textbf{关于图片:}}
\begin{enumerate}
    \item 请使用\texttt{graphicx}(而非\texttt{graphics})宏包
    \footnote{后者与前者属于同一个项目,但其功能相对于前者偏少,不便使用。}
    \item 核心命令:\texttt{\textbackslash{}includegraphics}
    \item 注意区分图片和浮动体,后者只是承载图片的容器!
    \item 图片处理的工作应交由外部编辑器完成
\end{enumerate}

\red{\textbf{关于表格:}}
\begin{enumerate}
    \item 科技论文及书籍,优先使用三线表(\texttt{booktabs}宏包)
    \item 大表格可使用转换软件(如\texttt{Excel2LaTeX})或在线转换网站(如{TablesGenerator}
    \footnote{网址:\texttt{http://www.tablesgenerator.com/latex\_tables}}
    )来可视化编辑
    \item 表格可能是\LaTeX 中最让人头疼的一部分之一,请多花时间操练
\end{enumerate}
\end{frame}
\begin{frame}{浮动体处理}
\begin{alertblock}{关于浮动体的警告}
\begin{enumerate}\kaishu
    \item {\heiti 不要使用依赖于图形放置位置的文本。} 使用如 “这幅图...”或“下面的图
    形...”等短语要求所指的图形需在固定位置。而像“图5...”这样的短语则允
    许图形出现在任意位置\footnotemark。
    \item {\heiti 放松。}一些使用者在发现图形没有十分准确的出现在他们所想要的位置
    时,往往非常着急。这没有必要,图形的放置是\LaTeX{}的工作,最好放松一些。
\end{enumerate}
\footnotetext{科技论文中往往这样做,无论这些论文是否使用\LaTeX 排版。因此,不必担心读者
对此会产生意见,接受过高等教育的人对此都是相当习惯的。}
\begin{flushright}
——引自《\LaTeXe 插图指南》P87
\end{flushright}
\end{alertblock}
\end{frame}

\section{数学公式}
\begin{frame}{数学公式}
\begin{enumerate}
    \item 基本环境(美元号,\texttt{equation(*)},\texttt{split},\texttt{align}……)
    \item 准备一张够用的符号表
    \item 公式字体与正文字体的区分
\end{enumerate}
\end{frame}

\section{基层语法}
\begin{frame}{基层语法}
\begin{enumerate}
    \item \red{\textbf{命令}(command)}:执行一系列复合的功能,可以有参数也可以没有参数。是\LaTeX
    发明的新概念。
        \begin{itemize}
            \item 使用\texttt{\textbackslash{}newcommand}创建新命令
            \item 使用\texttt{\textbackslash{}renewcommand}更新命令
        \end{itemize}
    \item \red{\textbf{环境}(environment)}:定制了样式、实现了一组特定功能的区域块,由一对
    \texttt{\textbackslash{}begin}和\texttt{\textbackslash{}end}命令给定
    \item \red{\textbf{控制序列}(control sequence)}:执行一系列功能,没有参数,对整个
    作用域
    \footnote{\textbf{作用域}可以具体为:(1)在离控制序列最近的一对大括号或一个环境内;(2)在以上范围的
    基础上,进一步缩小为控制序列以后的那一部分,即控制序列不能对其之前的部分生效。例如,命令“\texttt{\{
    我\{爱\{中\textbackslash{}bfseries 国\}\}\}中国爱我}”中就将只有最内大括号内、控制序列之后的“国”字被加粗。}
    生效。是\TeX 中原生的概念。
\end{enumerate}

\textit{关于底层语法,不妨到各个宏包、模板、文档类的源代码中去自行摸索。}
\end{frame}

%以下内容留到最后一次讲座
%\section{\LaTeX{}与 Markdown 交互}
%\section{钱院资料编写小组工作方式概述}


\end{document}


