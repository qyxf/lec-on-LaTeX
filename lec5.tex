\documentclass{beamer}
\usepackage{ctex}
\setCJKmainfont{思源宋体} % 设置缺省中文字体
\usepackage{fourier}     % 配置西文与数学字体

\hypersetup{pdfcreator=TeX,
pdfpagemode = FullScreen,
}

\mode<presentation>
{
  \usetheme{XJTU}
  \useinnertheme{rounded}
  \usefonttheme{serif}
}

\title[Tech Lec]{钱院学辅技术讲座}
\subtitle{第五讲:\LaTeX 总结与排版规范}

\author[xjtu-blacksmith]{黑山雁}

\institute[Xi'an Jiaotong University]
{
  西安交通大学
}

\date{\today}

\subject{技术,讲座}

\AtBeginSubsection[]
{
  \begin{frame}<beamer>{目录}
    \tableofcontents[currentsection,currentsubsection]
  \end{frame}
}

%\beamerdefaultoverlayspecification{<+->}


\begin{document}
\begin{frame}
\maketitle
\end{frame}

\begin{frame}{目录}
    \tableofcontents
\end{frame}

\section{\LaTeX 知识点归纳}

\section{宏包世界入境指南}
% comment: 学校教的编程语言没有把“求助他人”变成一种习惯,单纯考试?
% geometry, fancyhdr, hyperref
% graphicx, subfig, pgf/tikz
% amsmath, siunitx, physics
% indentfirst

\section{结束了?}
\subsection{你还可以去看些什么}
\subsection{开始搬砖}

\section{排版规范}
\subsection{西文排版规范}
\subsection{中文排版规范}
\subsection{\LaTeX 代码规范}

\end{document}


