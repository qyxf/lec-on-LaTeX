\documentclass[10pt]{beamer}

\hypersetup{pdfcreator=TeX,
pdfpagemode = FullScreen,
}

\usepackage{fontspec}
\usepackage[fontset=adobe]{ctex}
\usepackage{metalogo,texnames}
\setmainfont{Cambria} % 设置缺省英文字体

\mode<presentation>
{
  \usetheme{XJTU}
  \useinnertheme{rounded}
  \usefonttheme{serif}
}

\usepackage{multicol}
\usepackage{indentfirst}

\title[钱学森书院学业辅导中心]{\LaTeX 讲座}
\subtitle{第一部分:初识\TeX\&\LaTeX}

\author[黑山雁]{\texttt{黑山雁 xjtu-blacksmith}}
% {F.~Author\inst{1} \and S.~Another\inst{2}}

\institute[Xi'an Jiaotong University]
{\textbf{西安交通大学·钱学森书院学业辅导中心}}

\date
{\today}

\subject{LaTeX 讲座} % meta info

\begin{document}

\begin{frame}
  \titlepage
\end{frame}

\begin{frame}{目录}
  \tableofcontents
\end{frame}

\section{\LaTeX 是什么}

\subsection{一切的开始:\TeX 系统的诞生}
\begin{frame}{一切的开始:\TeX 系统的诞生}{“我只是做一个软件玩玩”}
\begin{itemize}
    \item \textbf{\red{高德纳}} \textit{Donald Knuth}
    \begin{itemize}
        \item \textbf{算法分析}的奠基人
        \item \textbf{TAOCP}:\textit{The Art of Computer Programming}
        \item \TeX, \MF 的开发者
        \item \textbf{文学编程}概念的提出者
    \end{itemize}
    \item \red{\TeX 的诞生(1978)}:一个古老的传说
    \begin{itemize}
        \item 对于出版社给TAOCP的低劣排版感到极度不适,“\textit{研究古今排版艺术精华}”,开发了\TeX 排版系统
        \item \TeX 系统最初是用\texttt{Pascal}写的,后来用\texttt{C}重写
        \item 输出的文件为\texttt{DVI}格式(Device Independent file format,类于机器码),供不同设备打印
        \item 对于字词断行、字体处理、数学公式排版等支持良好,已成为数学、物理、计算机学界文档排版的\textbf{事实标准}
    \end{itemize}
\end{itemize}
\end{frame}

\subsection{\LaTeX 与\TeX 的关系}
\begin{frame}{\LaTeX 与\TeX 的关系}{\LaTeX 使我们不需要直接操作\TeX 系统}
\begin{itemize}
    \item \red{\LaTeX 是什么}:根植于\TeX 系统
    \begin{itemize}
        \item \TeX 的上手难度较大,底层处理较多;文档结构不明晰
        \item 由Leslie Lamport开发(1983),目前的版本为\LaTeXe(离\LaTeX3还相当遥远)
        \item 大大简化了文档的撰写流程,是目前使用\TeX 的首选方案
        \item \LaTeX 本质上是一种\textbf{\TeX-Markup}(\textit{编译器报错往往是报}\TeX\textit{系统的底层错误})
    \end{itemize}
    \item \red{凭什么用\LaTeX}:为什么不用\texttt{Word}?
    \begin{itemize}
        \item 编程式排版,文档结构清晰,样式易控制,输出文档干净利索
        \item 排版质量高,“高端大气上档次”
        \item Markup语言,容易与其他格式相互转化(\texttt{markdown},\texttt{html},\texttt{doc})
        \item 对于数学公式、矢量图的良好支持
    \end{itemize}
\end{itemize}
\end{frame}

\subsection{一份\LaTeX 文档的典型结构}
\begin{frame}[fragile]
\frametitle{一份\LaTeX 文档的典型结构}

{\tiny
\begin{verbatim}
\documentclass[12pt,a4paper]{article}
\usepackage{ctex}
\usepackage[top=1cm,bottom=1.5cm,left=2cm,right=2cm]{geometry}
\title{黑山雁的个人介绍}
\author{黑山雁}
\date{\today}
\begin{document}
\maketitle
我叫\textbf{黑山雁},是加里敦大学的一年级新生。
我的专业是\textit{计算机科学与技术}。
我将从这里开始启航,从这里走向远方。
我感到非常激动。

以下简单地介绍一下我自己。
\section{个人背景}
我世居东海之滨,见惯了潮涨潮退,日出日落。...
\section{求学经历}
我先后在开关厂幼儿园、希望小学、第一中学就读。...
\section{性格特点}
我不是一个认真的人,但我认真起来不是人。...
\section{对未来发展的期待}
我的梦想是月入百万,早日走上人生巅峰。...
\end{document}
\end{verbatim}
}

\begin{itemize}\small
    \item \red{编译的必要元素}:\texttt{documentclass}, \texttt{document}环境
    \item \red{文档结构}:\texttt{maketitle}, \texttt{section}等
    \item \red{样式设置}:\texttt{textbf}, \texttt{textit}等
    \item \red{外部引用}:宏包(\texttt{ctex}),文档类,参数设置
\end{itemize}
\end{frame}

\subsection{\LaTeX 代码如何被编译}
\begin{frame}{\LaTeX 代码如何被编译}
\begin{itemize}
    \item \red{原生\LaTeX}:\texttt{tex->dvi(->ps/pdf)}
    \item \red{现代方法}:\texttt{tex->pdf} (PDF\LaTeX, \XeLaTeX, \LuaLaTeX)
    \item \LaTeX 不是独立的“编程语言”,编译时仍要返诸\TeX 系统:编译器报错机制的\textbf{固有缺陷}
    \item \red{先编译再可视}:\texttt{tex}代码文件不可直接可视化,需先生成\texttt{pdf}文档才能查看效果(\textit{“所思即所得” vs. “所见即所得”})
    \item \red{重要的相关文件}:不需要的话可以在编译后删除
    \begin{itemize}
        \item \texttt{\red{log}}:日志文件(和你在控制台看到的一样)
        \item \texttt{\red{toc}}:章节目录样式文件
        \item \texttt{\red{out}}:引用标记文件(书签、超链接)
        \item \texttt{\red{bbl}}:由\BIBTEX 生成的记录文件
    \end{itemize}
\end{itemize}
\end{frame}


\section{如何正确地学习\LaTeX}
\subsection{万事开头难:不要陷进这三个坑}
\begin{frame}{万事开头难:不要陷进这三个坑}{切勿自废武功}
\begin{enumerate}
    \item \red{请远离\texttt{CJK}宏包与$\mathbb{C}$\TeX 套装}
    \begin{itemize}
        \item \texttt{CJK}是十年前处理中文的方式
        \item $\mathbb{C}$\TeX 已经多年未更新,功能较为冗余
        \item 处理中文,优先使用\texttt{ctex}宏包或\texttt{xeCJK}宏包
    \end{itemize}
    \item \red{不要按\texttt{Word}的思路来学习\LaTeX}
    \begin{itemize}
        \item 常见误区:强制换行、更换字体、图文混排、浮动对象
        \item Meaningless questions: “如何在Word中使用\LaTeX?” “怎样在\LaTeX 中实现Word中的XX功能?”
        \item 请逐渐习惯\LaTeX 的思维方式——胸有成竹
    \end{itemize}
    \item \red{切勿花费过多精力于\LaTeX 的细枝末节之上}
    \begin{itemize}
        \item \TeX/\LaTeX 是近四十年前的发明,与现代程序设计原理有所冲突(事无巨细,力求完美)
        \item 四十年以来的层层累进,内容太多(网状分布),不指望能够马上学通
        \item 现代Markup语言已能使内容与样式充分分离
        \item “\textit{不要做}\LaTeX\textit{的斗牛犬}”,文档的内容最重要
        \item 使用\LaTeX 的最高境界:排好内容且不浪费时间
    \end{itemize}
\end{enumerate}
\end{frame}

\subsection{安装\TeX 发行版}
\begin{frame}{安装\TeX 发行版}{“工欲善其事,必先利其器”}
\begin{enumerate}
    \item \textbf{\TeX 发行版}
    \begin{itemize}
        \item \texttt{\red{TeX Live}}:TUG开发,跨平台,更新及时,\red{值得信赖}
        \item \texttt{\red{MikTeX}}:Windows专享,宏包安装较方便,\red{值得信赖}
        \item \texttt{\red{MacTeX}}:虽然我买不起苹果电脑,但是仍然\red{值得信赖}
        \item \red{$\mathbb{C}$\TeX}:虽然不推荐,但是还是可以用的,\red{习惯就行}
    \end{itemize}
    \item \textbf{编辑器选择}
    \begin{itemize}
        \item 专用免费编辑器:\texttt{TeXworks},\texttt{TeXStudio},\texttt{TeXmaker},\texttt{TeXshop}
        \item 专用收费编辑器:\texttt{WinEdt}(\textit{可以买一个})
        \item 通用文本编辑器:\texttt{Vim},\texttt{VS Code},\texttt{Sublime},\texttt{Atom},\texttt{Notepad++}
        \item 其他:\texttt{Notepad}(\textit{其实挺好的})
    \end{itemize}
    \item \textbf{忠告:}不要在编译环境上花费太多精力,顺手就行;一定要把宏包装齐(\texttt{full}),否则很有可能引起自闭
\end{enumerate}
\end{frame}

\subsection{关于\TeX 的编译引擎}
\begin{frame}{关于\TeX 的编译引擎}
\begin{enumerate}
    \item \TeX(\texttt{latex}):\texttt{tex->dvi->pdf}(需要其他工具)
    \item PDF\TeX(\texttt{pdflatex}):\texttt{tex->pdf}(不支持Unicode,西文首选)
    \item Lua\TeX(\texttt{lualatex}):\texttt{tex->pdf}(支持Unicode,但不稳定)
    \item \XeTeX(\texttt{xelatex}):\texttt{tex->xdv->pdf}(支持Unicode,中文首选)
    \item \BIBTEX(\texttt{bibtex}):输出参考文献
    \item \textbf{请使用\texttt{ctex}宏包或\texttt{ctexart}等文档类,配合\red{UTF-8编码}进行中文排版!}
\end{enumerate}
\end{frame}

\subsection{自学资料推荐}
\begin{frame}{自学资料推荐}{以下资料仅供入门使用}
\begin{enumerate}
    \item \textbf{\red{电子文档}}
    \begin{itemize}
        \item \texttt{lshort}:涵盖了所有常用的功能与理念,涉及范围较广,有对中文排版的介绍
        \item \texttt{A Primer}:经典,着重于\LaTeX 的基本功能,内容更详细
        \item \texttt{lnotes}:可供进阶阅读,对小白不完全友好
    \end{itemize}
    \item \textbf{\red{出版书籍}}
    \begin{itemize}
        \item \red{刘海洋 - \LaTeX\textit{入门}}:内容较新,覆盖面广,内容安排较合理
        \item \red{胡伟 - \LaTeXe\textit{完全学习手册}(第二版)}:内容较深,介绍详细,可以提升水平
        \item \red{陈志杰 等 - \LaTeX\textit{入门与提高(第二版)}}:内容已经过时(\texttt{CJK}方式),但仍然值得参考,有很多很好的points
    \end{itemize}
    \item \textbf{\red{其他}}:\textit{终身学习吧!}
    \begin{itemize}
        \item \texttt{LaTeXStudio}:\texttt{www.latexstudio.net} (有微信公众号和QQ群)
        \item 知乎(很多dalao的活跃场所;要适度)
        \item \texttt{Stack Exchange}:\texttt{tex.stackexchange.com}
        \item 宏包与文档类的自带\texttt{doc}(\texttt{texdoc}命令)
    \end{itemize}
\end{enumerate}
\end{frame}

\section{关于\LaTeX 公式}
\subsection{\LaTeX 公式的基本语法}
\begin{frame}{\LaTeX 公式的基本语法}
\begin{itemize}
    \item \textbf{两种基本环境(不编号)}:行内\texttt{\$...\$},行间\texttt{\$\$...\$\$}
    \item \textbf{基本要素}:独立性,上下标,作用域
    \begin{itemize}
        \item \red{独立性}:除上下标/分数以外,不同的符号之间无从属关系,从左至右排列
        \item \red{上下标}:上标\texttt{\^},下标\texttt{\_},可用于求和/积分/括号等
        \item \red{作用域}:用大括号\texttt{\{\}}合成多个符号与命令(如在上下标中)
    \end{itemize}
    \item \textbf{学习公式的最好方法}:自行练习\textbf{拆分}、\textbf{组合}公式!
    \item \textbf{一个简短的练习}:
    $$\lim_{n\to\infty}\int_{-\pi}^\pi\frac{n!\cdot2^{2n\cos\phi}}{\left|\prod\limits_{k=1}^n(2n\mathrm{e}^{i\phi}-k)\right|}\mathrm{d}\phi=2\pi$$
    \item \textbf{以上用到的命令}:\texttt{lim},\texttt{to},\texttt{infty},\texttt{int},\texttt{cdot},\texttt{frac},\texttt{prod}(\texttt{sum})
\end{itemize}
\end{frame}

\subsection{在哪里可以用\LaTeX 公式?}
\begin{frame}{在哪里可以用\LaTeX 公式?}{不提供\LaTeX 公式输入的论坛都是……}
\begin{enumerate}
    \item \red{允许使用\LaTeX 输入的地方}:
    \begin{itemize}
        \item 知识分享社区 - 知乎 \texttt{www.zhihu.com}
        \item 中文百科网站 - 百度百科 \texttt{baike.baidu.com}
        \item 国际百科网站 - 维基百科 \texttt{en.wikipedia.org}
        \item 英文数学论坛 - \texttt{artofproblemsolving.com/community}
        \item 知识搜索引擎 - Wolfram Alpha \texttt{www.wolframalpha.com}
        \item 博客平台 - CSDN \texttt{blog.csdn.net}
        \item 博客平台 - 博客园 \texttt{www.cnblogs.com}
    \end{itemize}
    \item \red{你还可以用}:
    \begin{itemize}
        \item 在线公式编辑 - CodeCogs \texttt{www.codecogs.com/latex/eqneditor.php}
        \item 公式扫描转换 - Mathpix Snipping \texttt{mathpix.com}
        \item 支持\LaTeX 的公式编辑器 - MathType \texttt{...}
        \item Word Ofiice 2003的公式编辑器
        \item 画图工具+手写板绘制公式
        \item 写在纸上然后拍照
        \item \_\^\$\{\}\~$\backslash$
    \end{itemize}
\end{enumerate}
\end{frame}


\section{作业}
\begin{frame}{第一次讲座的课后作业}{要么认真写,要么不写}
\begin{block}{自由排版练习}
\begin{itemize}
\item 请选择下列文档之一进行排版,格式转化、字体选择、页面样式设置、横排还是竖排、自己写/用工具转换还是百度搜均无所谓,唯一的要求是在排版后将\texttt{tex}及\texttt{pdf}文件交给我。我将提供相关的修改意见,帮助大家一同提高进步。
\item 写作业时,可以将原来的文本做一定删改(但不要添加),以减少工作量而体现排版效果。\textbf{另外,请大家不要将自己写好的东西发布到网上!!!}
\end{itemize}
\end{block}

\begin{enumerate}
    \item 费立涵同学贡献的《\textit{GRE备考指南V1.0}》(\texttt{docx})
    \item 我发布的一篇博客《\textit{连续函数与“有理”分析}》(\texttt{html})
    \item 鲁迅《\textit{魏晋风度及文章与药及酒之关系}》(\texttt{txt})
\end{enumerate}
\end{frame}

\end{document}


