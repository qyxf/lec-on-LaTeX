\documentclass[10pt]{beamer}

\hypersetup{pdfcreator=TeX,
pdfpagemode = FullScreen,
}

\usepackage{fontspec}
\usepackage[fontset=adobe]{ctex}
\usepackage{metalogo,texnames}
\setmainfont{Cambria}

\mode<presentation>
{
  \usetheme{XJTU}
  \useinnertheme{rounded}
  \usefonttheme{serif}
}

\usepackage{multicol}
\usepackage{indentfirst}

\title[钱学森书院学业辅导中心]{\LaTeX 讲座}
\subtitle{第三部分:Markdown简介与任务布置}

\author[黑山雁]{\texttt{黑山雁 xjtu-blacksmith}}

\institute[Xi'an Jiaotong University]
{\textbf{西安交通大学·钱学森书院学业辅导中心}}

\date{\today}

\subject{LaTeX 讲座}

\begin{document}

\begin{frame}
  \titlepage
\end{frame}

\begin{frame}{目录}
  \tableofcontents
\end{frame}

\section{关于\LaTeX 的最后建议}
\begin{frame}{关于\LaTeX 的最后一点内容}{真的再没有了}
\begin{alertblock}{警告}
{\kaishu 学完\LaTeX{}是不可能学完的,一辈子都不可能学完的。}
资料的亮点在于其内容,排版样式只能在内容精良的前提下助其锦上添花。
\begin{enumerate}
    \item \textbf{\red{纯文本}}:用\texttt{notepad}都行,不需要任何\LaTeX{}技术
    \item \textbf{\red{含公式文本}}:会一点\LaTeX{}公式语法即可,不需要系统学习\LaTeX{}
    \item \textbf{\red{正文排版}}:若已掌握\texttt{part2}中提到的所有内容,并已反复操练,
    烂熟于心,则便已完全足够
    \item \textbf{\red{模板与宏包设计}}:人外有人,天外有天,学习是一辈子的事情
\end{enumerate}
\end{alertblock}
\end{frame}

\section{Markdown简介}
\subsection{什么是Markdown}
\begin{frame}{什么是Markdown}{先Mark再说}
\begin{itemize}
    \item \red{Markdown}:文本标记语言(\textbf{Mark-up})
    \item \red{基本原理}:用最简单的命令、符号在纯文本中指定必要的样式
\end{itemize}

\begin{center}\color{XJTUdarkred}
\texttt{*Hello*} -> \textit{Hello}; \texttt{**Hello**} -> \textbf{Hello};
\end{center}

\begin{itemize}
    \item \red{语言特性}:与\texttt{html}兼容;不限定编译方式(\textit{可依据实际需要
    定制格式与样式});不设定语法规范(\textit{流派众多});可以内嵌\LaTeX{}公式(
    \textit{需要外部支持})
    \item \red{功能特性}:内容与样式完全分离,节省撰写和排版时间;易于与其他文本/富文本格式
    相互转化;简洁、自由,上手快,人见人爱
    \item \red{流行趋势}:越来越多的网站、论坛、邮箱乃至代码编辑器支持或默认Markdown为
    标准文本格式
\end{itemize}

\end{frame}

\subsection{自学Markdown指南}
\begin{frame}{自学Markdown指南}{这可能会是你见过的最简单的“编程语言”}
作为一门简单易上手的工具,请各位同学(\textit{特别是对\LaTeX{}不感兴趣,又想贡献内容和
参与部分排版工作的同学})自学Markdown。
\begin{enumerate}
    \item \red{实践中学习}(English):\texttt{https://www.markdowntutorial.com}
    \item \red{轻松入门}(English):\texttt{https://guides.github.com/features/mastering-markdown}
    \item \red{本国语言}:\texttt{https://www.appinn.com/markdown/basic.html}
    \item \red{进阶指南}:\texttt{http://www.markdown.cn}
\end{enumerate}

\begin{block}{语录}
{\kaishu
世界上最简单而最富有成就感
\footnote{成就感是虚幻的,其实不会有人去看你写的教程。}
的事情,是写一份Markdown入门教程。更富有成就感的事情是:用Markdown写Markdown教程。

\begin{flushright}
——智叟
\end{flushright}
}
\end{block}
\end{frame}

\subsection{Markdown与\LaTeX{}的对应关系}
\begin{frame}{Markdown与\LaTeX{}的对应关系}{这是一个单射}
\begin{enumerate}
    \item \red{文档结构}:章节标题(\texttt{\textbackslash{}section -> \#})
    \item \red{文本控制}:粗斜体(\texttt{\textbackslash{}textbf -> **...**}),抄录
    (\texttt{\textbackslash{}verb|...| -> `...`}),列表(\texttt{itemize}与
    \texttt{enumerate}),引用(\texttt{quote -> >...}),脚注(\texttt{\textbackslash{}
    footnote -> [\^\ ...]}),超链接(\texttt{\textbackslash{}href -> []()})
    \item \red{分段机制}:完全一样(?!)
    \item \red{图表处理}:图片插入(\texttt{\textbackslash{}includegraphics -> ![]()}),
    表格绘制(\texttt{tabular -> ---|---|---})
    \item \red{数学公式}:Markdown支持在正文中用\texttt{\$\$} 等符号或环境插入公式(
    取决于具体编译器的实现机制)
\end{enumerate}

\begin{quotation}
学好\LaTeX{}语法,走遍一切Mark-up都不怕!
\end{quotation}
\end{frame}

\subsection{Markdown流派简述}
\begin{frame}{Markdown流派简述}{“天下分久必合,合久必分”}
\begin{itemize}
    \item \red{原生Markdown (\texttt{markdown.pl})}:不支持表格,不支持自动链接,不
    支持……转换规则不甚严格
    \item \red{CommonMark} \includegraphics[width=0.04\textwidth]{pic/Markdown.png}:
    一套比较公认的Markdown规范格式,转换规则严格
    \item \red{GFM (GitHub Flavored Markdown)}:衍生自CommonMark,删除线、自动链接、
    代码高亮、表格语法
    \item \red{MultiMarkdown}:脚注、参考文献、元数据、公式、TOC、导出增强……功能很多;
    自带一套多格式转换器
    \item \red{Pandoc's Markdown}:Pandoc软件自定义的Markdown语法,可微调,与GFM基本兼容,
    亦支持许多MultiMarkdown中的功能
\end{itemize}

\begin{block}{提示}
\textbf{不建议}在分辨各流派的“\textit{特有语法}”方面花费太多精力,以自己用的顺手的语法架
构为准。可到BabelMark
\footnote{网址:\texttt{https://babelmark.github.io}}
网站测试自己的Markdown代码在不同语法下效果如何。
\end{block}
\end{frame}

\subsection{Markdown与其他文档格式交互}
\begin{frame}{Markdown与其他文档格式交互}{越简单,越混乱}
\begin{itemize}
    \item Markdown本就是一种用于导出的纯文本格式(To \texttt{html})
    \item 一般的Markdown引擎都是将Markdown转换为\texttt{html} 块
    \item 专用编辑器、Markdown插件往往支持导出为\texttt{pdf},\texttt{epub}等个别常
    见格式,样式难以调整
    \item Pandoc与MultiMarkdown等软件可实现多格式互转,但可读性、兼容性堪忧
\end{itemize}

\begin{block}{提示}
除了Markdown到HTML的单向转换,以及在严格的语法和样式约束下撰写的Markdown到样式固定的
\texttt{pdf}等的单向转换,其余的格式转化都是\textbf{不稳定、不可靠}的,只能作为一种
\textit{辅助手段}加快排版的进度、减少重复劳动。
\end{block}
\end{frame}

\section{学研部任务布置与工作规范}
\subsection{组织架构与任务分工}
\begin{frame}{组织架构与任务分工}\footnotesize
\begin{itemize}
    \item \red{\textbf{组织架构}}
    \begin{enumerate}\footnotesize
        \item \red{编写组}:负责提供、初步整理、收集归档资料,并以可能的方式整理为初稿;根据
        具体的任务、项目,进一步分化为若干从属的工作小组
        \item \red{排版组}:负责将编写组提供的稿件转写为规范的\LaTeX{}/Markdown文档,及时更新
        GitHub仓库中的版本;与编写组之间不做显著区分,可从属于各项目的工作小组
        \item \red{发行组}:负责印刷品最终的样式调整、模板设计与维护,以及作品的印刷、发布
    \end{enumerate}
    \item \red{\textbf{运行机制}}
    \begin{enumerate}\footnotesize
        \item \red{工作小组}:每项中等及以上工作量的多人项目必须成立工作小组,设定时间表,
        充分讨论的同时及时跟进工作(配合好GitHub中的相关功能)
        \item \red{分层次运行}:\red{编写组}(\textit{润色内容})->\red{排版组}(
        \textit{规范格式})->\red{发行组}(\textit{统一样式})三层次,互相促进,确保作
        品的质量
        \item \red{明确任务}:提前规划相关事项,及时跟进任务进度,多项任务同时进行
    \end{enumerate}
    \item \red{\textbf{任务分工}}
    \begin{enumerate}\footnotesize
        \item {\kaishu 如果你只会用Word,或不愿花费精力做排版工作}:\red{编写组}(允许提供手写稿)
        \item \textit{如果你只是酷爱敲公式、敲回车}:\red{编写组}或\red{排版组}(抄写员)
        \item {\kaishu 如果你是\LaTeX{}高手}:\red{排版组}及\red{发行组}
        \item \textit{如果你有好的内容}:请提供详细的资料/策划,我们将为你组织\red{工作小组}
        \item {\kaishu 如果你有好的创意/美工设计}:请直接提出
    \end{enumerate}
\end{itemize}
\end{frame}

\subsection{撰写与排版工作规范}
\begin{frame}{撰写与排版工作规范}
\begin{block}{\LaTeX{}文档规范}\scriptsize
\begin{enumerate}
    \item 尽量使用默认样式进行排版效果的测试,如有特殊需求应当注明
    \item 使用新宏包前应检查其版本,确保其版本和年代较新
    \item 减少缩进的使用
    \item 确保章节的划分已能达到一般出版物的水平
    \item 图片的调整和表格的样式设定等,不要过于苛求,此工作留由发行组进行
    \item 避免强制换行、手工加距离(\texttt{\textbackslash{}vspace})、非必要换页
    等不自然、不能自适应样式调整的操作
    \item 充分利用粗斜体、常见特殊环境等基本功能来优化文本效果,不得使用复杂盒子、色彩
    调整、自定义字体
\end{enumerate}
\end{block}

\begin{alertblock}{Markdown文档规范}\scriptsize
\begin{enumerate}
    \item 使用Markdown作为最终格式的前提:内容简单,近于纯文本,不含公式的交叉引用,
    不需要图文混排
    \item 暂以GFM为标准语法规范,提交文档前应通过GitHub的网页端进行测试
    \item 不得使用\texttt{html}标签
    \item 使用代码块即可,不要使用代码高亮
\end{enumerate}
\end{alertblock}

\centering\kaishu 排版组出产的文档,限定为\LaTeX{}或Markdown格式。

\end{frame}

\subsection{奖励制度与资料分发}
\begin{frame}{奖励制度与资料分发}
\begin{exampleblock}{工时发放标准}\footnotesize
\begin{enumerate}
    \item \textit{章节式资料编写}:每章总计发放6-8工时,根据质量确定,按照参与者的贡献量
    再行分配
    \item \textit{单篇资料编写}:2000字以下发放4-6工时,5000字以上按字数和质量发放8-12工时
    \item \textit{排版}:按照\LaTeX{}代码行数,每200行发放1工时
\end{enumerate}

其他未提及的情况,暂不制定标准,根据实际情况再行商议。
\end{exampleblock}

\begin{alertblock}{资料如何分发}\footnotesize
\begin{enumerate}
    \item \red{电子版}:\LaTeX{}生成\texttt{pdf}文件后,发布在百度网盘上供同学下载
    (\textit{需加密为不得打印的版本})
    \item \red{纸质版}:对优质且需求量大的作品,由学辅付费印刷并向同学们发放
    \item \red{源代码}:根据内容提供者的意愿,发布在GitHub上或交由其自行处理
    \item \red{宣传活动}:针对好的作品,发布相关的海报、推送,或开展赠书活动,让钱院乃
    至全校的各位同学均能看到
\end{enumerate}
\end{alertblock}
\end{frame}

\subsection{近期需要筹备的事项}
\begin{frame}{近期需要筹备的事项}
\begin{enumerate}
    \item 组织架构的逐步定型
    \item 各项工作
    \footnote{目前正在进行的项目包括:大物题解、流力题解、已有资料排版和模板设计。}
    的稳步推进(可参见\texttt{github/qyxf}的Project)
    \item 关于各班级课程情况的统计,以及对目前流行资料的调查
    \item 第一期杂志和第一版新生指南的筹备
\end{enumerate}

\begin{alertblock}{我该做些什么?}
\begin{enumerate}
    \item 第9周周三前,给学研部三位负责人之一提交一份自述,字数不限,应包括:
    \textit{我会/擅长/希望做什么,我想参与哪项工作(我要提出什么工作),我能在资料编写方面投入多大精力}
    \item 学习Markdown,有余力的情况下学习Git与GitHub的基本使用方法
    \item 积极贡献资料,参与编写工作
\end{enumerate}
\end{alertblock}
\end{frame}

\begin{frame}
\vfill
\begin{center}\Huge\itshape
Thank you!
\end{center}
\vfill
\end{frame}

\end{document}


